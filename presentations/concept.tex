\documentclass{beamer}
\usepackage[T1]{fontenc}
\usepackage[utf8]{inputenc}
\usepackage[ngerman]{babel}
%\usepackage{paralist}
%\useoutertheme{infolines} 
\usepackage{graphicx}
\usepackage{hyperref}
\usepackage{listings}
\usepackage{color}
\usetheme{Warsaw}
\usecolortheme{crane}
\pagenumbering{arabic}
\def\thesection{\arabic{section})}
\def\thesubsection{\alph{subsection})}
\def\thesubsubsection{(\roman{subsubsection})}
\setbeamertemplate{navigation symbols}{}
\graphicspath{ {src/} {/home/jim/Pictures/} }

\definecolor{mygreen}{rgb}{0,0.6,0}
\definecolor{mygray}{rgb}{0.5,0.5,0.5}
\definecolor{mymauve}{rgb}{0.58,0,0.82}

\lstset{ %
  backgroundcolor=\color{white},   % choose the background color; you must add \usepackage{color} or \usepackage{xcolor}
  basicstyle=\footnotesize,        % the size of the fonts that are used for the code
  breakatwhitespace=false,         % sets if automatic breaks should only happen at whitespace
  breaklines=true,                 % sets automatic line breaking
  captionpos=b,                    % sets the caption-position to bottom
  commentstyle=\color{mygray},    % comment style
  deletekeywords={},            % if you want to delete keywords from the given language
  escapeinside={\%*}{*)},          % if you want to add LaTeX within your code
  extendedchars=true,              % lets you use non-ASCII characters; for 8-bits encodings only, does not work with UTF-8
  keepspaces=true,                 % keeps spaces in text, useful for keeping indentation of code (possibly needs columns=flexible)
  keywordstyle=\color{blue},       % keyword style
  language=PHP,                 % the language of the code
  morekeywords={class, function, return, protected, public, private, const, static, new, extends, namespace, null},            % if you want to add more keywords to the set
  numbers=left,                    % where to put the line-numbers; possible values are (none, left, right)
  numbersep=5pt,                   % how far the line-numbers are from the code
  numberstyle=\tiny\color{mygray}, % the style that is used for the line-numbers
  rulecolor=\color{black},         % if not set, the frame-color may be changed on line-breaks within not-black text (e.g. comments (green here))
  showspaces=false,                % show spaces everywhere adding particular underscores; it overrides 'showstringspaces'
  showstringspaces=false,          % underline spaces within strings only
  showtabs=false,                  % show tabs within strings adding particular underscores
  stepnumber=2,                    % the step between two line-numbers. If it's 1, each line will be numbered
  stringstyle=\color{mygreen},     % string literal style
  tabsize=2,                       % sets default tabsize to 2 spaces
  title=\lstname                   % show the filename of files included with \lstinputlisting; also try caption instead of title
}

\hypersetup{
	pdfauthor=Jim Martens,
	pdfstartview=Fit
}

\expandafter\def\expandafter\insertshorttitle\expandafter{%
	\raggedleft \insertframenumber\,/\,\inserttotalframenumber\;}

\begin{document}
\author{Jim 2martens, 2martens Development Group}
\title{2martens Web Platform}
\date{June 2014}

	% Introduction
	\begin{frame}
		\titlepage
	\end{frame}
	
	% Motivation
	\begin{frame}{Motivation}
		\begin{itemize}
			\item Why do we need it?
			\item Such a thing already exists...
			\item If noone has done it yet, it has a reason...
			\item I am using already X, Y and Z and it works...
		\end{itemize}
		
		Well...
	\end{frame}
	
	\begin{frame}{Motivation}
		\begin{itemize}
			\item it's state-of-the-art
			\item it's extendable
			\item it's functional
			\item it's modular
			\item it's testable
			\item it's free
		\end{itemize}
		
		Enough of the buzz words, let's get serious
	\end{frame}	
	
	% Contents
	\begin{frame}{Agenda}
		\tableofcontents
	\end{frame}
	
	\section{Basics}
	\begin{frame}{Beginning}
		In the beginning was Symfony2. And it was good. There was Drupal as CMS but no real forum. Nor were there applications that were intercompatible from the get-go.
		
		The main reason being that Symfony2 is mostly used as a base for highly customized applications.
	\end{frame}
	
	\begin{frame}{Goal}
		Deliver a web-platform that provides core community functionality similar to the WoltLab Community Framework (WCF) and central features that are desperately needed for every modern web application.	
	\end{frame}
	
	\begin{frame}{Core Features}
		\begin{itemize}
			\item Administrator Control Panel (ACP)
			\item User and Group system
			\item Package system
			\item Style system
			\item Project system
		\end{itemize}
	\end{frame}
	
	\section{Details}
	\subsection{Approach}
	\begin{frame}{Comparisons}
		\begin{table}
			\begin{tabular}{l|r}
				Our feature & 3rdParty \\
				\hline
				ACP & Sonata Admin Bundle \\
				Package system & Composer \\
				User and Group system & FOSUserBundle \\
				Style system & Twitter Bootstrap/LESS
			\end{tabular}
		\end{table}
		
		\centering Why don't we use existing software?
	\end{frame}
	
	\begin{frame}{Common Policy}
		\begin{itemize}
			\item seperate bundles
			\item common core bundle for shared code
			\item combined functionality
		\end{itemize}
		
		Multiple levels of bundles:
		\begin{itemize}
			\item Core-level (e.g. common non-PHP 3rdParty libraries)
			\item Basic utility bundles (reusable outside the web-platform, few dependencies, e.g. UploadBundle)
			\item High-level bundles (bundles providing core features like package system)
			\item Application-level bundles (ACP) 
		\end{itemize}
	\end{frame}
\end{document}

