\RequirePackage{pdf14}
\documentclass{beamer}
\usepackage[T1]{fontenc}
\usepackage[utf8]{inputenc}
\usepackage[ngerman]{babel}
%\usepackage{paralist}
%\useoutertheme{infolines} 
\usepackage{lmodern,textcomp}
\usepackage{graphicx}
\usepackage{hyperref}
\usepackage{listings}
\usepackage{color}
\usetheme{Warsaw}
\usecolortheme{twomartens}
\pagenumbering{arabic}
\def\thesection{\arabic{section})}
\def\thesubsection{\alph{subsection})}
\def\thesubsubsection{(\roman{subsubsection})}
\setbeamertemplate{navigation symbols}{}
\graphicspath{ {src/} {/home/jim/Pictures/} }

\definecolor{mygreen}{rgb}{0,0.6,0}
\definecolor{mygray}{rgb}{0.5,0.5,0.5}
\definecolor{mymauve}{rgb}{0.58,0,0.82}

\definecolor{craneorange}{RGB}{61,61,61}
\definecolor{craneblue}{RGB}{255,255,255}

\lstset{ %
  backgroundcolor=\color{white},   % choose the background color; you must add \usepackage{color} or \usepackage{xcolor}
  basicstyle=\footnotesize,        % the size of the fonts that are used for the code
  breakatwhitespace=false,         % sets if automatic breaks should only happen at whitespace
  breaklines=true,                 % sets automatic line breaking
  captionpos=b,                    % sets the caption-position to bottom
  commentstyle=\color{mygray},    % comment style
  deletekeywords={},            % if you want to delete keywords from the given language
  escapeinside={\%*}{*)},          % if you want to add LaTeX within your code
  extendedchars=true,              % lets you use non-ASCII characters; for 8-bits encodings only, does not work with UTF-8
  keepspaces=true,                 % keeps spaces in text, useful for keeping indentation of code (possibly needs columns=flexible)
  keywordstyle=\color{blue},       % keyword style
  language=PHP,                 % the language of the code
  morekeywords={class, function, return, protected, public, private, const, static, new, extends, namespace, null},            % if you want to add more keywords to the set
  numbers=left,                    % where to put the line-numbers; possible values are (none, left, right)
  numbersep=5pt,                   % how far the line-numbers are from the code
  numberstyle=\tiny\color{mygray}, % the style that is used for the line-numbers
  rulecolor=\color{black},         % if not set, the frame-color may be changed on line-breaks within not-black text (e.g. comments (green here))
  showspaces=false,                % show spaces everywhere adding particular underscores; it overrides 'showstringspaces'
  showstringspaces=false,          % underline spaces within strings only
  showtabs=false,                  % show tabs within strings adding particular underscores
  stepnumber=2,                    % the step between two line-numbers. If it's 1, each line will be numbered
  stringstyle=\color{mygreen},     % string literal style
  tabsize=2,                       % sets default tabsize to 2 spaces
  title=\lstname                   % show the filename of files included with \lstinputlisting; also try caption instead of title
}

\hypersetup{
	pdfauthor=Jim Martens,
	pdfstartview=Fit
}

\expandafter\def\expandafter\insertshorttitle\expandafter{%
	\raggedleft \insertframenumber\,/\,\inserttotalframenumber\;}

\begin{document}
\author{Jim 2martens}
\title{2martens Web Platform}
\date{\today}

\begin{frame}
    \titlepage
\end{frame}

\begin{frame}
    \tableofcontents
\end{frame}

\section{Einführung}
\begin{frame}{Fragen...}
  \begin{itemize}
    \item Was?
    \item Warum?
    \item Wofür?
    \item Wieviel?
    \item Wann?
  \end{itemize}
\end{frame}

\begin{frame}{...Antworten}
  \begin{itemize}
    \item Webplattform
    \item gibt's so noch nicht, möchte ich haben
    \item viele Dinge
    \item 0€ + Freie Software
    \item When it's done
  \end{itemize}
\end{frame}

\section{Zeitlinie}
\begin{frame}{Ursprung}
  \begin{itemize}
    \item lange Zeit mit Woltlab Community Framework (WCF) gearbeitet
    \item unterstützt keine automatischen Tests
    \item hat aber Paketsystem, Mehrsprachigkeit für usergenerierte Inhalte und Stylesystem
    \item fehlende Dokumentation
  \end{itemize}
\end{frame}
\begin{frame}{Ursprung}
  \begin{itemize}
    \item Symfony entdeckt
    \item in vielen Punkten deutlich besser
    \item Problem: kein Paketsystem, kein ACP, kein Stylesystem, keine Mehrsprachigkeit für usergenerierte Inhalte
    \item Idee: das beste vom WCF mit der Power von Symfony kombinieren (Arbeitstitel: Symfony WCF)
    \item das war Anfang 2014
  \end{itemize}
\end{frame}
\begin{frame}{Recherche}
  \begin{itemize}
    \item Symfony hauptsächlich für Custom-Webseiten eingesetzt
    \item keine Plattform vorhanden (z.B. Drupal nutzt Symfony, ist aber ein CMS)
    \item einige Projekte hier und da, aber keines mit dem Ziel von mir
  \end{itemize}
\end{frame}
\begin{frame}{Start}
  \begin{itemize}
    \item Ende Mai 2014: Namenswahl
    \item PHPUnconf2014: Entscheidung nächstes Jahr etwas zeigen zu können
    \item einzelne Orgacommits im Dezember 2014
    \item März 2015: intensives Design der 2martens Web Platform, hauptsächlich des Core Bundles
    \item seit dem Entwicklung in unregelmäßigen Abständen
  \end{itemize}
\end{frame}

\section{Design}
\subsection{High-level}
\begin{frame}{Überblick}
  \begin{itemize}
    \item Core Bundle
    \item Language Bundle
    \item Content Bundle
    \item Media Bundle
    \item Upload Bundle
    \item Mail Bundle
    \item Frontend Essentials Bundle
    \item Community Bundle
    \item Poll Bundle
    \item Conversation Bundle
    \item Sanction Bundle
  \end{itemize}
\end{frame}
\begin{frame}{Prinzip}
  \begin{itemize}
    \item alles außer Core Bundle optional
    \item sog. Feature-Bundles können aber andere Feature-Bundles voraussetzen
    \item jedes Feature Bundle implementiert einen Funktionsbereich umfassend
    \item optionale Abhängigkeiten: z.B. Forum installiert ohne Poll Bundle;
          Installation von Poll Bundle im ACP -> Umfragen jetzt im Forum möglich
    \item Mix and match statt monolithische Riesenplattform
    \item erweiterbar
    \item storage agnostic, soweit möglich
  \end{itemize}
\end{frame}
\subsection{Core Bundle}
\begin{frame}{Überblick}
  \begin{itemize}
      \item ACP
      \item Paketsystem
      \item Optionssystem
      \item User- und Gruppensystem
      \item Stylesystem
      \item Suchsystem
      \item Menüsystem
      \item ``Cronjob''system
    \end{itemize}
\end{frame}

\begin{frame}{ACP}
  \begin{itemize}
    \item funktionsfähiges ACP
    \item erweiterbar
    \item zusätzliche Bundles integrieren sich in ACP
  \end{itemize}
\end{frame}

\begin{frame}{Paketsystem}
  \begin{itemize}
    \item basierend auf Composer
    \item neue Bundles werden als Abhängigkeit zur Distribution hinzugefügt
    \item daher kein eigenes Dependency Management
    \item Composer-Plugin, um neue Bundles zu integrieren
    \item Custom AppKernel, um dynamisch Bundle ergänzen zu können
    \item Unterstützung für initiale Installation
  \end{itemize}
\end{frame}

\begin{frame}{User- und Gruppensystem}
  \begin{itemize}
    \item Gruppen sind weit mehr als Ansammlung von Roles
    \item Gruppenoptionen über boolean-Werte hinaus
    \item erweiterbar für neue Bundles
    \item benutzt FOSUserBundle
    \item einige systemrelevante Gruppen (admin, user, everyone)
  \end{itemize}
\end{frame}

\begin{frame}{Deep Dive Gruppenoptionen}
  \begin{itemize}
    \item Gruppenoptionen in .yml-Dateien persistiert (app/config/group/<group>.yml)
    \item Boolean-Optionen werden ergänzend als Roles in der Gruppe in der Datenbank gespeichert
    \item Optionen in drei Bereiche unterteilt: acp, mod, frontend
    \item Roles setzen sich so zusammen: ROLE\_<AREA>\_<BUNDLE CATEGORY>\_<OPTION NAME>
    \item z.B. ROLE\_ACP\_TWOMARTENS.CORE\_ACCESS
    \item Role pro Gruppe (z.B. ROLE\_ADMIN für admin-Gruppe)
    \item jede Gruppe hat publicName und roleName (z.B. Administrators und ADMIN)
  \end{itemize}
\end{frame}

\begin{frame}{Demo}
  \centering
  DEMO
\end{frame}

\section{Abschluss}
\begin{frame}{Ausblick}
  Ziel:
  \begin{itemize}
    \item Forum, CMS, ... in 2WP installiert
    \item wirkt wie aus einem Guss
    \item Verbesserungen an anwendungsunabhängigen Features betreffen alle
    \item kein Vendor lock-in (Dateninkompatibilität)
    \item hochqualitative Website ohne viel Geld und ohne Programmierkenntnisse aufsetzen können
    \item flächendeckend bessere Websites mit ingesamt höheren Mindeststandards
  \end{itemize}
\end{frame}

\begin{frame}{Kontakt}
  \begin{itemize}
    \item Website: 2martens.de
    \item GitHub: github.com/2martens
    \item Twitter: @2martens
  \end{itemize}
\end{frame}

\begin{frame}{Q \& A}
  \centering
  Fragen...
\end{frame}

\end{document}

